\documentclass[man, floatsintext]{apa7}

\usepackage{lipsum}
\usepackage{tabularray}
\usepackage{multirow}
\usepackage[american]{babel}

\usepackage{csquotes}
\usepackage[style=apa,backend=biber,doi=false,url=false]{biblatex}
\addbibresource{bibliography.bib}
\usepackage{float}
\usepackage{graphicx}
\graphicspath{{./figures}}
\title{Modelling the effects of information gathering on social decision making}
\shorttitle{Modelling the effects of information gathering on social decision making}

\author{Damon Hayhurst}
\date{\today}

\makeatletter
\renewcommand{\maketitle}{
	\begin{titlepage}
		\centering
		\vspace*{0.4in}
		{\Huge \bfseries \@title \par}
		\vspace{0.2in}
		{\LARGE Damon Hayhurst \par}
		\vspace{0.3in}
		{\Large Project Report Submitted as\par}
		{\Large partial fulfilment for the\par}
		{\Large Degree of Cognition and Computation,\par}
		\vfill
		{\Large Birkbeck, University of London\par}
		{\Large \@date \par}
		\vfill
		{\Large
			\begin{center}
				I confirm that this write up is my own work and does not involve\par
				plagiarism as defined in the module information provided
			\end{center}
		}
		{\Large Signed and dated\par}
		\vspace{1in}
	\end{titlepage}
}
\makeatother

% Custom abstract environment to center the text
\renewenvironment{abstract}{
	\clearpage
	\null
	\vspace*{0.4in}
	\begin{center}
		This is the abstract
	\end{center}
	
	\vfill\null
	\clearpage
}

\begin{document}

\maketitle

% Project execution which includes: originality of ideas, independence, ability to act on advice and time and effort expended

\abstract

% Abstract which includes: completeness, clarity and succinctness

\section{Introduction}

% Introduction which includes: breadth of the review, understanding of the literature, description of aims or hypotheses, organization and structure and clarity

% What is the breadth of the review?
% The review is about investigating the strategies employed when making decisions in a social context, using the messenger game as a framework within which to investigate such strategies. The breadth therefore comes from looking at investigations into the ways in which search strategies have been shown to influence decison making.
% Intertemporal Choice
% link between strategy and choice
% Bounded reality
% Process Tracing


%The background to the study should be provided here including a review of relevant literature and theory and clear statements about the aims and objectives of the study.

\subsection{Background}

Are decisions made in our own self-interest influenced by the manner in which we form our conclusions? Previous work in the study of decision making has shown that the way in which we gather information can influence the final outcome. In a social setting, self interested and other facing motives have been shown to have their value tracked independently by specific neural circuitry. Using mouse tracking data from a similar experiment we ask whether the manner in which we gather information can predict which one of these motives we are likely to pursue. 

\subsection{Aims/Objectives}

The aims of this study is to gain an insight into how the effects of information gathering influence our decision making in a social context.  Using process tracing, we hope to observe and characterise strategies that are used in order to arrive at these decisions. Such an approach hopes to shed light further on the processes that underlie decision making in human behaviour.


\begin{itemize}
	\item Ask whether information gathering strategy can be used a predictor in social decision making
	\item Understand whether perceived monetary outcomes to self and another have an affect on the way we gather information
	\item Determine if information gathering strategy in social decision making reflects an individual preference
\end{itemize}

\subsection{Understanding decision making in humans}

Present day understanding of decision making in humans takes it's inspiration from Herbert A Simon's approach of Bounded Rationality. Where prior understanding suggested that human beings act as  rational agents always seeking to maximise self interest, Bounded Rationality suggested that humans capability to act in a perfectly rational manner was constrained by internal and external factors.

%Herbert A Simon first outlined the idea that human behaviour in decision making was not to be defined by the decisions taken but that of the information surrounding the decision. This concept of 'Bounded Rationality' suggested an approach that took consideration of the contextual information regarding the decision as well as the computational capacities available to the decision maker. The concept sought to defy prior notions that human behaviour in decision making could be understood through a global notion of homo economius; Where decisions taken by agents were always perfectly rational, seeking to maximise their own self interests, and, instead, suggested that a human's capability to be perfectly rational was constrained by internal and external constraints.

Bounded Rationality places as much of an emphasis on the processing of information during the decision making as the decision itself. Pivotal to the concept, it was suggested that human's mental capacity to make choices is subject to heuristics or mental simplifications that contend with it's ability to be a perfectly rational agent but aid it's ability to make decisions in the face of increasing complexity (\cite{payneTaskComplexityContingent1976}).  Simon outlined Satisficing as a key mechanism employed in the face of decisional complexity. Such a model of process attempts to define the appropriate option amongst choices as being the first option to surpass pre conceived threshhold. Rather than cope with the cognitive overhead of comparing every option against all of the others in a perfectly rational manner, humans will opt for the first choice that is deemed sufficient and satisfactory (\cite{simonRationalChoiceStructure1956d}, \citeyear{simonBehavioralModelRational1955}).

Early work into the mechanisms that perpetrate the notion of Bounded Rationality revealed the necessity for capturing process data. The Priority Heuristic attempted to explain choices within the context of simple monetary gambles using an ordered list of criteria. While the mechanism was shown to be significant at the aggregate level, the model failed to capture differences at the individual level. Further research into the manifestation of the Priority Heuristic, using process data, uncovered that in actual fact the underlying process did not mimic the priority ordering set out by the heuristic. The data, however, revealed that choice could be predicted significantly by the number of transitions into and out of a choice. The uncovering of such mechanisms highlights the need for processing data in explaining how humans make decisions. Process data can provide the explanatory power for decision making differences and inform the development of heuristics that can be applied at the individual level ((\cite{brandstatterPriorityHeuristicMaking2006}, (\cite{willemsenVisitingDecisionFactory2011})).

%It is important to note that it is widely believed that human's decision making strategy is constructed contingent on the task demands and individual preferences. Decision making strategies have been shown to vary widely based on the number of choices at hand and the task complexity (\cite{payneWalkingScarecrowInformation2004}). Human's have a 'toolbox' of strategies which can be utilised in the cwhen it comes to information gathering and the choice reflects task demands and individual preferences. Such strat

\subsection{Capturing the decision making process }

Process tracing is a methodology used to gain insight into the decision-making process by extracting data that reveals the underlying cognitive mechanisms. This involves various techniques, ranging from simple verbal protocol analysis, where participants articulate their thoughts aloud, to advanced methods like neural tracing using MRI machines (\cite{fordProcessTracingMethods1989}). The goal of these methods is to capture detailed data from different stages of the decision-making process and use that information to inform and develop process models.

Studying decision-making involves analyzing different stages, such as problem identification, information gathering, alternative evaluation, and decision selection. Each stage requires specific methods of data extraction to effectively capture the nuances of decision-making. For instance, in the information gathering stage, researchers might employ eye-tracking to see where attention is focused, or they might use tools like MouseLabWeb to observe how individuals search for and process information. The choice of instrumentation is crucial, as insights into decision making are limited by the resolution and representation of the output data. Efforts must be made to consolidate these representations with process models in order to enhance their development and applicability.

\subsubsection{MouseLabWeb}

MouseLabWeb (and offline counterpart, MouseLab) is one such development in process tracing that fosters an environment fit for observing the information gathering stage in decision making. Utilising the dynamic properties of Javascript in the context of a web page, MouseLabWeb describes an interactive framework that tracks users' attentional focus on decision-making elements as they engage with different options. MouseLabWeb experiments consist of a browser page containing adjacent pairs of hidden view boxes. Each pair represents an option and the make up of each pair is a hidden view box containing a piece of information relevant to that particular option. Positioning the mouse cursor over one of these boxes reveals the concealed information. The mouse cursor movements are tracked and then stored on a trial by trial basis. Such data can be used to observe where a participant's focus lies. The simplistic design of a MouseLab experiment reduces the cognitive load on the participant. Affording them simple cursor movements instead of lookup to memory (\cite{MouselabWEBa}). 

\subsection{Interpreting the data}

Interpreting results from a MouseLab experiment involves discerning attributional qualities from a given trace. \citeauthor{costa-gomesCognitionBehaviorNormalForm2001} define two events in a process trace that can provide insight into the underlying characteristics of a trace. Occurence, being that a particular piece of information has occurred within a process trace and therefore, that piece of information can be considered as part of the overall strategy underlying the trace. An absence of an occurrence of a particular piece of information suggests a strategy that is made without considering the full picture and could be conceived as impulsive. And Adjacency, if two adjacent pieces of information occur sequentially within a trace then it means a comparison has likely to have taken place. An absence of adjacency could represent a disregard for certain aspects underlying the decision such as in social preference based choices, where the adjacent pairs show to be payoffs related to you or the other participant. An absence of adjacency in this case suggests motives that are wholly self concerned. The decisional lookups and their contextual information can together help attach a type to a specific process trace (\cite{costa-gomesCognitionBehaviorNormalForm2001}).




Defining the type of search strategy employed and being able to predict that information has shown success before.

Reeck and all found success in showing a link between the underlying type of a search strategy and a decisional outcome. Using the Intertemporal Choice Task as the vehicle for observing search strategy, wherein participants had to choose between small reward at an earlier date or a larger one at a late date. They observed that search strategies which exhibited a patient type, as in had qualities that showed large quantities of adjacency, were significantly more likely to result in outcomes that represented the more patient choice. 

\subsection{Understanding decision making as a cognitive process}

Our cognition can be seen as a continuous dynamic system





% occurence and adjacency
% intertemporal choice task
% cognition dynamic system




% paragraph on contingency



\section{Methods}

% Methods which includes: completeness, justification of choices/decisions made, organization and structure, clarity, reference to ethics

% This section is composed of several sub-sections and should give sufficient information to enable the reader to know exactly what was done:
%
%Participants (in projects using human participants).This will be skipped if you are not using human participants. Please specify in this section that ethical approval was obtained and include the ethics approval number and the title of the ethics application.
%
%Materials and Stimuli (all stimuli involved in your project, including software, questionnaires, programmes, interview schedules, for example). Feel free to include pictures of your stimuli. Describe everything.
%
%Design and Procedure (this section should be detailed enough that someone not familiar with your project should be able to run it themselves/understand how you recruited participants, collected data and so on). 

A secondary analysis was carried out from the results of pre existing experiment. The primary stimuli, of which, was a variation of The Message Game that was carried out within a MouseLabWeb environment. 

\subsubsection{The Message Game}
The Message Game is a form of experimental game that models a particular complexity faced by individuals in pro-social environments. The game uses honesty to illustrate how decision making can be affected by value judgements that pit our own self interest against our altruistic sensitivities. 
The game involves two roles. The Sender and the Receiver. The Sender must send a message to the Receiver informing them of a particular box to open from a choice of four. Each of the boxes contains differing amounts of money for the Sender and the Receiver should that box be chosen. The Sender can ultimately see what's inside each of the boxes in terms of how much money they will receive versus the amount that the Receiver will. Only the Receiver opens the box, however. A message is sent by the Sender to the Receiver with words to the affect of 'Box X is best for you'. The dishonesty comes from the Sender being able to suggest a box that might not necessarily represent the truth and instead picking a box that contains a larger amount of money for themselves and less money for the Receiver. With the Receiver unable to see the quantity of money they potentially missed, they must decide whether to trust the Sender's recommendation and open the suggested box.

This variation has the focus placed entirely on the Sender. Two of the boxes are blank meaning that the Receiver has no choice but to pick the Sender's recommendation or risk a 2 in 3 chance of getting no monetary income. The participant plays as the Sender for the majority, however, does get to have a go at playing at the Receiver role only to understand the dilemma faced. The participant then spends 80 trials in the Sender role where the process data is tracked. The position of the monetary rewards in the boxes are then altered on a trial by trial basis so that decisions are encouraged to be deliberate and do not succumb to laziness. 

The values held in each of the two boxes vary on a trial by trial basis. Each of the two boxes consists of two values representing a profit for the Sender and Receiver. These boxes always contain differing amounts for the Receiver and Sender. The box with the higher amount for the Receiver, if chosen, represents the Sender telling the truth to the Receiver. Likewise, the box with the lower amount represents the Sender making a dishonest claim to the Receiver about the potential profits that could've been earned.

Being that the trials are carried out in a MouseLabWeb environment, we then correspond each of the two amounts contained within the boxes to Areas of Interest (AOIs). The AOI representing what amount the Sender receives for the truth telling box is SELF TRUE and the one representing the amount for the Receiver is OTHER TRUTH. AOIs corresponding to the amounts for the dishonest box are SELF LIE for what the Sender receives and OTHER LIE for the receiver. Theses AOIs are the pieces of information hidden from view that can only be revealed by hovering a mouse cursor over the panel. Horizontally adjacent lookups correspond to comparisons of amount awarded to one particular role and vertical adjacency compares values awarded for the both roles for a given option. The vertical position of the AOIs swaps as well in order to encourage deliberate choices.

With the values and positioning of each AOI altering on a trial basis, one can gain an insight into the process and circumstances that inhibit a participant from acting in a rational manner. The game theoretic viewpoint of suggests that the Sender should always choose to act in their own self interest by picking the option that represents the maximum utility to themselves ie. the box with the most value to them. As ultimately their are no repercussions for lying. Using human participants and recording process data, we can observe when a human's altruistic sensitivities take over.

% OVer experiement was found no difference mouse vs vision
% Look into the idea of dwelling on something by looking at it.

\subsection{Procedure}

Results for the experiment were spread across two CSVs which resulted in a collective total of 89 participants. Rows represented a single trial for each participant. The rows provided information such as the values that were in each AOI, the selected box number, reaction time and process trace.

Since trial order was randomised, an index number was assigned to each trial. That number was based upon every unique set of values contained within the AOIs. Trials that shared the same amounts were assigned the same index value. See Figure \ref{fig:Gains} and \ref{fig:Losses}. The trials could then be aggregated across participants based on the profits and losses earned by the Sender and Receiver per trial.

The trials were then grouped per participant and were assessed under a set of criteria that might warrant their exclusion. A breakdown of these can be found in the Results section.

Each remaining trial had a process trace consisting of a semi-colon delimited set of timestamps and coordinates representing the position of the mouse cursor at any given time during the trial.  By corroborating this information with the stated participants screen width and the per trial position of each box and containing values, it could be calculated whether a particular timestamp represented a cursor position that was inside one of the four AOIs. From this, a timeline could be built showing durational occurrences of any of the four AOIs within the process trace.

\subsection{Average Analysis}

Using the newly created timeline, the average dwell for each of the four AOIs was calculated per trial. Along with the number of transitions between adjacent AOIs per trial. 

\subsection{Cluster Analysis}

An extra measure of similarity was calculated between all timelines in order to 


\subsubsection{Creating time series sequences}

\subsubsection{Measuring the distance between trials}

\subsubsection{Clustering}


\section{Results/Findings}

\subsection{Trial and Participant Quantities}
\label{subsec:quantities}
The processing of data removed a proportion based on criteria that represented a lack of engagement with the experiment. six participants were removed for not finishing it. 20 were removed for telling the truth over 95\% the time and a further 14 participants were removed for recommending the blank boxes to the receiver in over five percent of the trials.  Interestingly, no participants were removed for lying over 95\% of the time; despite this being the approach favoured in game theory.  A further 40 trials were removed from the dataset because they recorded decision making times that were three standard deviations away from the mean. An approach also taken by \citeauthor{reeckSearchPredictsChanges2017b}, suggesting a similar lack of engagement with the task at hand.

Irregularities and missing data that represented a failure in process tracing were also removed from the dataset. 10 participants were removed for having not recorded mouse coordinates in over 85\% of their trials and a further 108 trials removed from the remaining valid set because of similar lack of mouse coordinates. Irregularities in some of the coordinate data meant that AOIs could not identified through measurement. 42 trials were removed because of this and one participant also, that failed to record screen width data. If such impairments in the dataset had been included by merely estimating their missing qualities, it would have compromised the overall integrity. Omitting them highlights the essential role of accurate process tracing.

In further regards to accurate process tracing, some consideration was taken towards events that lasted under a threshold of 200ms. This process accounts for jittery movements by the participant and periods of dwell that do not amount to the recognised minimum for reading hidden text in object recognition (\cite{dicarloHowDoesBrain2012}) and MouseLab literature (\cite{willemsenVisitingDecisionFactory2011}). Once all events in each trial accounted for this threshold, a further 103 trials were removed from the total citing an absence of any remaining valid dwell events within AOIs. 

After accounting for engagement criteria and irregularities, the number of valid trials was recorded for each participant and a further six participants were excluded from the analysis. The number of valid trials for each of these participants represented a quantity that was less than three quarters of the intended 80 trials for each participant. Given, the ordering of each trial was randomised, the set of valid trials for each of these participants could represent biases in likelihood to lie relative to the whole data set.

From the original pool of 89 participants, only 32 remained after all removals. An overall data set that contained 2489 trial instances each with a valid process trace.

\begin{figure}[H]
	\includegraphics[width=\linewidth]{../plots/RESPONSE/NTrialsByPID.png}
	%\figurenote{This is a great figure.}
	\label{fig:NTrialsByPID}
	\caption{Bar graph representing number of trials in the final valid data set for each participant.}
\end{figure}

\subsection{Descriptives}

\subsubsection{Lie Percentage}

Across trials the propensity to lie displayed a high degree of variance. The average likelihood to lie on any given trial was 38\% $(N = 2489)$. Across participants, the propensity to lie was measured with a 22\% (Q1) - 51\% (Q3) interquartile range for a given individual. Across instances of unique trial, where unique is defined by the specific quantities pertained to in the boxes, the range was 0 - 90\% with an interquartile range of 16\% (Q1) - 59\% (Q3) for any given trial.

\begin{figure}[H]
	\includegraphics[width=\linewidth]{../plots/RESPONSE/PIDPercentLiesPlot.png}
	%\figurenote{This is a great figure.}
	\label{fig:PIDPercentLiesPlot}
	\caption{Bar graph representing the percentage amount of lies across participants}
\end{figure}

\begin{figure}[H]
	\includegraphics[width=\linewidth]{../plots/RESPONSE/TRIALIDPercentLies.png}
	%\figurenote{This is a great figure.}
	\label{fig:TRIALIDPercentLies}
	\caption{Bar graph showing the percentage of lies for each trial condition.}
\end{figure}

\subsubsection{Dwell Time Distribution}

Reaction time (or the time it takes to make a decision) showed a large amount of range. The average reaction time was 6513 ms $(SD = 3050$ms$)$ with an interquartile range of 4393 (Q1) - 7871 ms (Q3).

\begin{figure}[H]
	\includegraphics[width=\linewidth]{../plots/Dwell/RTDistPlot.png}
	%\figurenote{This is a great figure.}
	\label{fig:RTDistPlot}
	\caption{Distribution of reaction time ie. time taken to make a decision.}
\end{figure}

The amount of spent dwelling on one of the four AOIs on the screen was considered. The AOIs, SELF LIE ($M = 470$ms, $SD = 428$ms), SELF TRUE ($M = 420$ms, $SD = 343$ms), OTHER LIE ($M = 571$ms, $SD = 423$ms) and OTHER TRUTH ($M = 604$ms, $SD = 472$ms) all recorded means within 150ms of one another. A large proportion of the trials reported no dwell time for at least one of the given AOIs. Both SELF TRUE and SELF LIE were absent from the dwell timelines of 450 trials, representing an omission rate of 18\%. OTHER TRUE and OTHER LIE were omitted from 221 and 240 respectively representing around 10\% each, of all trials.

\begin{figure}[H]
	\caption{Distribution of average dwell time for each AOI. The distribution only accounts for average dwell time of each AOI that were under 1000ms. Each AOI also had a proportion of average dwell times that were 0, which were also omitted.}
	\includegraphics[width=\linewidth]{../plots/Dwell/DwellDistPlot.png}
	%\figurenote{This is a great figure.}
	\label{fig:DwellDistPlot}
\end{figure}

\begin{figure}[H]
	\caption{Distribution of number of transitions between AOIs across trials.}
	\includegraphics[width=\linewidth]{../plots/Dwell/N_TransitionsDistPlot.png}
	%\figurenote{This is a great figure.}
	\label{fig:NTransitionsDistPlot}
\end{figure}

The number of transitions is marked by where a participant would move the cursor from one AOI to another was also observed across all trials. The mean number of transitions between AOIs was ($M = 5.6$, $SD = 4.1$) with an interquartile range of 3 (Q1) - 8 (Q3).


\subsection{Task Analysis}

\subsubsection{Lie Percentage}

\begin{figure}[H]
	\caption{Bar chart showing the amount gained by the Sender should they choose to lie in the corresponding trial.}
	\includegraphics[width=\linewidth]{../plots/TrialIndex/Gains.png}
	%\figurenote{This is a great figure.}
	\label{fig:Gains}
\end{figure}

Each of the trials involved giving differing amounts to the sender and the receiver based on whether the sender decided to lie or not. The average net gain to the sender was 12 $(SD = 13)$ per trial. The median was 8 and had an interquartile range of 4 (Q1) - 17 (Q3). The average loss to the receiver was 11 $(SD = 9)$ with a median of 8 and an interquartile range of 4 (Q1) - 15 (Q3).

\begin{figure}[H]
	\caption{Bar chart showing the amount lost by the Receiver should the Sender choose to lie in the corresponding trial.}
	\includegraphics[width=\linewidth]{../plots/TrialIndex/Losses.png}
	%\figurenote{This is a great figure.}
	\label{fig:Losses}
\end{figure}

The percentage of senders that chose to lie in the experiment increased as the net gain to the sender did.  Senders lied the least when they incurred a net loss by lying. Only lying 6\% $(N = 63)$ of the time. From net gains below 10 to above 10 the percentage who chose to lie increased, from 24\% $(N = 1336)$ to 55\% $(N = 1153)$. When gains were above 20, the percentage who chose to lie in this task scenario grew to 70\% $(N = 560)$ and when they were above 30 the amount who decided to lie showed further increase but to a lesser extent 76\% $(N = 250)$.

\begin{figure}[H]
	\includegraphics[width=\linewidth]{../plots/RESPONSE/NetGainLie.png}
	\caption{Percentage who chose to lie based on the amount gained by lying.}
%	\figurenote{+10 gain represents the proportion who chose to lie for all amounts above 10 and also applies for +20 and +30.}
	\label{fig:NetGainLie}
\end{figure}

A significant difference in the proportion of lies was observed across trials where the sender gained less than 10 and when they received more than 10, $\chi^2(1,$ $N=2489) = 248,$ $p<.001$. 

\begin{figure}[H]
	\includegraphics[width=\linewidth]{../plots/RESPONSE/NetLossLie.png}
	\caption{Percentage of senders who chose to lie based on how much the receiver would lose.}
%	\figurenote{+10 gain represents the proportion who chose to lie for all amounts above 10 and also applies for +20 and +30.}
	\label{fig:NetLossLie}
\end{figure}

Senders were less likely to lie when the losses to the receiver were great. When losses were under 10,  41\% $(N = 1340)$ of senders chose to lie. With losses greater than 10, 55\%  $(N = 1153)$ chose to lie through to losses greater than 20 where the percentage of senders who chose to lie showed a notable increase, to 70\%  $(N = 560)$. When losses to the receiver were larger than thirty the amount of senders was 30\% $(N = 150)$

The difference in lie percentage between trials where losses to the receiver were over 10 and trials where losses to the receiver were under 10 was shown to be significant $\chi^2(1,$ $N=2027) = 45,$ $p<.001$ as well as in trials where losses to the receiver were over and under 30 $\chi^2(1,$ $N=2489) = 4.2,$ $p=.042$.

\subsubsection{Dwell Time}

\begin{figure}[H]
	\includegraphics[width=\linewidth]{../plots/RESPONSE/AvgDwellPerGain.png}
	\caption{Average Dwell Time for each AOI comparison between trials where net gain to the Sender was less than 10 and trials where the net gain was more than 10}
	%	\figurenote{+10 gain represents the proportion who chose to lie for all amounts above 10 and also applies for +20 and +30.}
	\label{fig:AvgDwellPerGain}
\end{figure}

\begin{table}[H]
	\centering
	\begin{tabular}{|c|p{1.5cm}|p{2cm}|p{1.5cm}|p{2cm}|p{2cm}|p{1.5cm}|}
		\hline
		\multirow{2}{*}{} & \multicolumn{2}{c|}{<10 Gain to Sender} & \multicolumn{2}{c|}{>10 Gain to Sender} & \multicolumn{2}{c|}{T-test result} \\ \cline{2-7}
		& $M$ (ms) &$SD$ (ms) & $M$ (ms) & $SD$ (ms) & $t(2487)$ & $p$ \\ \hline
		SELF LIE& 463 & 441 & 478 & 413 & 0.84 & .798  \\ \hline
		SELF TRUE & 430 & 354 & 408 & 329 & -1.6 & .460  \\ \hline
		OTHER LIE & 555 & 398 & 590 & 448 & 2.0 & .173 \\ \hline
		OTHER TRUTH & 598 & 455 & 611 & 492 & 0.66 & .511 \\ \hline
	\end{tabular}
	\vspace{0.3cm}
	\caption{Comparison of mean and standard deviation of dwell times for each AOI across trials where the net gain to Sender was less than 10 versus trials where the net gain was more than 10. (FDR Corrected)}
	\label{tab:NetGainDwell}
\end{table}


Table \ref{tab:NetGainDwell} shows that there was an no significant difference between the average dwell time of the four AOIs across trials where the net gain to the Sender was less than 10 versus when it was more than 10.

\begin{figure}[H]
	\includegraphics[width=\linewidth]{../plots/RESPONSE/AvgDwellPerLossPlot.png}
	\caption{Average Dwell Time for each AOI comparison between trials where net loss to the Receiver was less than 10 and trials where the net loss was more than 10}
	%	\figurenote{+10 gain represents the proportion who chose to lie for all amounts above 10 and also applies for +20 and +30.}
	\label{fig:AvgDwellPerLoss}
\end{figure}

\begin{table}[H]
	\centering
	\begin{tabular}{|c|p{1.5cm}|p{2cm}|p{1.5cm}|p{2cm}|p{2cm}|p{1.5cm}|}
		\hline
		\multirow{2}{*}{} & \multicolumn{2}{c|}{<10 Loss to Receiver} & \multicolumn{2}{c|}{>10 Loss to Receiver} & \multicolumn{2}{c|}{T-test result} \\ \cline{2-7}
		& $M$ (ms) &$SD$ (ms) & $M$ (ms) & $SD$ (ms) & $t(2491)$ & $p$ \\ \hline
		SELF LIE& 475 & 417 & 478 & 413 & 0.15 & .880  \\ \hline
		SELF TRUE & 418 & 328 & 408 & 329 & -0.71 & .476  \\ \hline
		OTHER LIE & 566 & 410 & 590 & 448 & 1.4 & .224 \\ \hline
		OTHER TRUTH & 594 & 441 & 611 & 492 & 0.92 & .511 \\ \hline
	\end{tabular}
	\vspace{0.3cm}
	\caption{Comparison of mean and standard deviation of dwell times for each AOI across trials where the net loss to the Receiver was less than 10 versus trials where the net loss to the Receiver was more than 10. (FDR Corrected)}
	\label{tab:NetLossDwell}
\end{table}

Similarly, in Table \ref{tab:NetLossDwell}, there was no significant difference between the dwell times in trials where the loss to the Receiver was above 10 versus trials where it was below 10.

\begin{figure}[H]
	\includegraphics[width=\linewidth]{../plots/RESPONSE/AvgDwellPerGain30.png}
	\caption{Average Dwell Time for each AOI comparison between trials where net gain to the Sender was less than 30 and trials where the net gain was more than 30}
	%	\figurenote{+10 gain represents the proportion who chose to lie for all amounts above 10 and also applies for +20 and +30.}
	\label{fig:AvgDwellPerGain30}
\end{figure}

\begin{table}[H]
	\centering
	\begin{tabular}{|c|p{1.5cm}|p{2cm}|p{1.5cm}|p{2cm}|p{2cm}|p{1.5cm}|}
		\hline
		\multirow{2}{*}{} & \multicolumn{2}{c|}{<30 Gain to Sender} & \multicolumn{2}{c|}{>30 Gain to Sender} & \multicolumn{2}{c|}{T-test result} \\ \cline{2-7}
		& $M$ (ms) &$SD$ (ms) & $M$ (ms) & $SD$ (ms) & $t(2487)$ & $p$ \\ \hline
		SELF LIE& 519 & 486 & 465 & 421 & -1.9 & .235  \\ \hline
		SELF TRUE & 399 & 329 & 422 & 345 & 1.0 & .476  \\ \hline
		OTHER LIE & 610 & 424 & 567 & 422 & -1.5 & .224 \\ \hline
		OTHER TRUTH & 582 & 428 & 607 & 477 & 0.79 & .511 \\ \hline
	\end{tabular}
	\vspace{0.3cm}
	\caption{Comparison of mean and standard deviation of dwell times for each AOI across trials where the net gain to the Sender was less than 30 versus trials where the net gain was more than 30. (FDR Corrected)}
	\label{tab:NetGainDwell30}
\end{table}

Across trials where the net gain to Sender and the net loss to the Receiver was at larger quantities there was found to be no significant difference in dwell times.

\begin{figure}[H]
	\includegraphics[width=\linewidth]{../plots/RESPONSE/AvgDwellPerLossPlot30.png}
	\caption{Average Dwell Time for each AOI comparison between trials where net loss to the Receiver was less than 30 and trials where the net loss was more than 30}
	%	\figurenote{+10 gain represents the proportion who chose to lie for all amounts above 10 and also applies for +20 and +30.}
	\label{fig:AvgDwellPerLoss30}
\end{figure}

\begin{table}[H]
	\centering
	\begin{tabular}{|c|p{1.5cm}|p{2cm}|p{1.5cm}|p{2cm}|p{2cm}|p{1.5cm}|}
		\hline
		\multirow{2}{*}{} & \multicolumn{2}{c|}{<30 Loss to Receiver} & \multicolumn{2}{c|}{>30 Loss to Receiver} & \multicolumn{2}{c|}{T-test result} \\ \cline{2-7}
		& $M$ (ms) &$SD$ (ms) & $M$ (ms) & $SD$ (ms) & $t(2487)$ & $p$ \\ \hline
		SELF LIE & 471 & 431 & 453 & 379 & 0.51 & .817  \\ \hline
		SELF TRUE & 421 & 345 & 399 & 307 & 0.76 & .476 \\ \hline
		OTHER LIE & 571 & 421 & 571 & 422 & 0.01 & .990 \\ \hline
		OTHER TRUTH & 601 & 470 & 649 & 507 & -1.2 & .511 \\ \hline
	\end{tabular}
	\vspace{0.3cm}
	\caption{Comparison of mean and standard deviation of dwell times for each AOI across trials where the net gain to the Sender was less than 30 versus trials where the net gain was more than 30. (FDR Corrected)}
	\label{tab:NetLossPerDwell30}
\end{table}

\subsubsection{Number of Transitions}

\begin{figure}[H]
	\includegraphics[width=\linewidth]{../plots/RESPONSE/NTransitionPerGain.png}
	\caption{Comparison of average number of transitions between trials where the net gain to the Sender was less than 10 and trials where the net gain was more than 10}
	%	\figurenote{+10 gain represents the proportion who chose to lie for all amounts above 10 and also applies for +20 and +30.}
	\label{fig:NTransitionPerGain}
\end{figure}

The difference in number of transitions proved not to be significant when comparing trials where the net gain to the Sender is above $(M = 5.5\%$, $SD = 4.1\%)$ or below 10 $(M = 5.6\%$, $SD = 4.1\%)$, $t(2487)=-0.5$, $p=.617$.

\begin{figure}[H]
	\includegraphics[width=\linewidth]{../plots/RESPONSE/NTransitionPerLossPlot.png}
	\caption{Comparison of average number of transitions between trials where net loss to the Receiver was less than 10 and trials where the net loss was more than 10}
	%	\figurenote{+10 gain represents the proportion who chose to lie for all amounts above 10 and also applies for +20 and +30.}
	\label{fig:NTransitionPerLoss}
\end{figure}

Similarly the difference in number of transitions showed to be insignificant when comparing trials where the Receiver made a loss of above $(M = 5.5\%$, $SD = 4.1\%)$  or below 10 $(M = 5.6\%$, $SD = 4.1\%)$ from lying, $t(2491)=-0.66$, $p=.617$.

\begin{figure}[H]
	\includegraphics[width=\linewidth]{../plots/RESPONSE/NTransitionPerGain30.png}
	\caption{Comparison of average number of transitions between trials where the net gain to the Sender was less than 30 and trials where the net gain was more than 30}
	\label{fig:NTransitionPerGain30}
\end{figure}

Across trails where the net gain to sender was above $(M = 5.1\%$, $SD = 3.6\%)$ or below 30  $(M = 5.6\%$, $SD = 4.2\%)$  there showed to be no significant difference, $t(2487)=1.9$, $p=.241$.

\begin{figure}[H]
	\includegraphics[width=\linewidth]{../plots/RESPONSE/NTransitionPerLossPlot30.png}
	\caption{Comparison of average number of transitions between trials where the net loss to the Sender was less than 30 and trials where the net loss was more than 30}
	\label{fig:NTransitionPerLoss30}
\end{figure}

Where the net loss to the receiver was above $(M = 5.1\%$, $SD = 3.6\%)$ and below 30 $(M = 5.1\%$, $SD = 3.6\%)$ there was no significant difference found, $t(2487)=1.3$, $p=.376$.

\section{Cluster Analysis}

Hierarchical clustering was used to fit suitable clusters based on the pairwise distance measure calculated between trials using Dynamic Time Warping. Using the Pseudo F statistic and a cluster limit of 20, it was deemed that two clusters represented the best degree of separation and internal cohesiveness. The two clusters, cluster 1 and 2 were made up of 1006 and 1483 trials, respectively.


\subsubsection{Lie Percentage}

\begin{figure}[H]
	\caption{Average percent lies across clusters formed from hierarchical clustering using Euclidean distance measure.}
	\includegraphics[width=\linewidth]{../plots/ALLTRIAL/PercentLies.png}
	%\figurenote{This is a great figure.}
	\label{fig:PercentLiePerCluster}
\end{figure}

There was a significant difference in the proportion of lies across clusters $\chi^2(1,$ $N=2489) = 24,$ $p<.001$.

\subsubsection{Dwell Times}

\begin{figure}[H]
	\caption{Average dwell time across clusters for each AOI.}
	\includegraphics[width=\linewidth]{../plots/ALLTRIAL/DwellTimes.png}
	%\figurenote{This is a great figure.}
	\label{fig:DwellTimesPerCluster}
\end{figure}

Each of the AOIs showed significant differences across clusters, as seen in Table \ref{tab:DwellPerCluster}.

\begin{table}[H]
	\centering
	\begin{tabular}{|p{1.4cm}|p{1cm}|p{1cm}|p{1cm}|p{1cm}|p{1cm}|p{1cm}|p{1cm}|}
			\hline
			\multirow{2}{*}{} & \multicolumn{2}{c|}{Cluster 1} & \multicolumn{2}{c|}{Cluster 2} & \multicolumn{3}{c|}{Between} \\ \cline{2-8}
			& $M$ (ms) &$SD$ (ms) & $M$ (ms) & $SD$ (ms)  & $t$ & d.o.f. & $p$   \\ \hline
			\small{SELF LIE}& 551 & 392 & 415 & 345 & 8.1 & 2321 & <.001  \\ \hline
			\small{SELF TRUE} & 530 & 309 & 345 & 345& 14 & 2305  &  <.001 \\ \hline
			\small{OTHER LIE} & 633 & 396 &529 & 435 & 6.0 & 2287 &  <.001  \\ \hline
			\small{OTHER TRUTH} & 679 & 452 & 554 & 479 & 6.6 & 2237 &  <.001 \\ \hline
		\end{tabular}
	\vspace{0.3cm}
	\figurenote{d.o.f., in this case, is shorthand for degrees of freedom.}
	\caption{Mean, standard deviation and t tests for the average dwell time of each AOI across participant clusters (FDR corrected).}
	\label{tab:DwellTimesPerCluster}
\end{table}


\subsubsection{Number of Transitions}

\begin{figure}[H]
	\includegraphics[width=\linewidth]{../plots/ALLTRIAL/NTransitions.png}
	\caption{Average number of transitions calculated per cluster for each participant.}
	%	\figurenote{+10 gain represents the proportion who chose to lie for all amounts above 10 and also applies for +20 and +30.}
	\label{fig:NTransitionsPerCluster}
\end{figure}

The average number of transitions across clusters also showed significant differences. With cluster 1 showing a mean level of 8.5 $(SD = 4.1)$ and cluster 2 showing a smaller average of 3.5 $(SD = 2.6)$.

\begin{figure}[H]
	\includegraphics[width=\linewidth]{../plots/ALLTRIAL/NTrialsByPID.png}
	\caption{The number of trials per cluster for each participant. Clusters were calculated using hierarchical clustering based on Euclidean distance.}
	%	\figurenote{+10 gain represents the proportion who chose to lie for all amounts above 10 and also applies for +20 and +30.}
	\label{fig:NTrialsByPIDPerCluster}
\end{figure}


It was observed that every participant had trials belonging to each of the cluster memberships, as seen in Figure \ref{fig:NTrialsByPIDPerCluster}. The mean amount of trials for a given participant that belonged to cluster 1 was 31 trials $(SD = 18)$ with a 12 (Q1) - 45 (Q3) trial interquartile range. Cluster 2 had a mean of 46 trials $(SD = 19)$ per participant and a 32 (Q1) - 64 (Q3) trial interquartile range. 




% Results/Findings which includes: analyses performed, reporting of analyses, use of graphs/tables/diagrams, organization and structure, and clarity
% analyses performed:
% RT, Dwell Times
% Gains
% DTW 

%\subsection{Participant Analysis}
%
%An analysis was performed on the distance data extracted from the time series analysis at the participant level. The results of which, are taken from averaging the distance between all of the trials of a single participant against another. Through hierarchical clustering and silhouette analysis, the participants were separated in to two clusters.
%
%\begin{figure}[H]
%	\includegraphics[width=\linewidth]{../plots/PID/DistanceMatrix.png}
%	\caption{Matrix showing the similarity between participants taken from their mean euclidean distance measure shared with one and another. The matrix is sorted by proximity with the two clusters of participants outlined.}
%	%	\figurenote{+10 gain represents the proportion who chose to lie for all amounts above 10 and also applies for +20 and +30.}
%	\label{fig:DistanceMatrix}
%\end{figure}
%
%\subsubsection{Lie Percentage}
%
%\begin{figure}[H]
%	\includegraphics[width=\linewidth]{../plots/PID/PercentLiesByPID.png}
%	\caption{Participants separated by hierarchical distance clustering and ordered by percentage lies.}
%	%	\figurenote{+10 gain represents the proportion who chose to lie for all amounts above 10 and also applies for +20 and +30.}
%	\label{fig:PercentLiesByPIDCluster}
%\end{figure}
%
%Using pairwise t-tests and calculating the mean statistic across clusters, it was deemed that the two clusters and their participant memberships showed to have a difference in lie percentage that was not significant $t(30)=1.4$, $p=.181$ (FDR corrected). Cluster 1 had a mean lie percentage of 40\% $(SD = 49\%)$ and cluster 2 had a mean of 23\% $(SD = 42\%)$. 
%
%\subsubsection{Dwell Time}
%
% \begin{figure}[H]
% 	\includegraphics[width=\linewidth]{../plots/PID/DwellTimes.png}
% 	\caption{Measures of average dwell time across participant clusters segmented by hierarchical distance clustering.}
% 	%	\figurenote{+10 gain represents the proportion who chose to lie for all amounts above 10 and also applies for +20 and +30.}
% 	\label{fig:DwellTimesByPIDCluster}
% \end{figure}
%
%\begin{table}[H]
%	\centering
%	\begin{tabular}{|p{1.4cm}|p{1cm}|p{1cm}|p{1cm}|p{1cm}|p{1cm}|p{1cm}|p{1cm}|p{1cm}|p{1cm}|p{1cm}|}
%		\hline
%		\multirow{2}{*}{} & \multicolumn{4}{c|}{Cluster 1} & \multicolumn{4}{c|}{Cluster 2} & \multicolumn{2}{c|}{Between} \\ \cline{2-11}
%		& $M$ (ms) &$SD$ (ms) & $t(378)$ & $p$ & $M$ (ms) & $SD$ (ms) & $t(112)$ & $p$ & $t(6)$ & $p$ \\ \hline
%		\small{SELF LIE}& 456 & 423 & 0.79 & .182 & 573 & 455 & -1.6 & .169 & 0.68 & .187  \\ \hline
%		\small{SELF TRUE} & 407 & 346 & 0.58 & .138 & 512 & 307 & -1.3 & .293 & 0.40 & .191  \\ \hline
%		\small{OTHER LIE} & 555 & 417 & -0.29 & .258 & 693 & 444 & -2.9 & .103 & -0.60 & .231 \\ \hline
%		\small{OTHER TRUTH} & 571 & 461 & -0.67 & .213 & 846 & 487 & -2.7 & .064 & -0.98 & .138 \\ \hline
%	\end{tabular}
%	\vspace{0.3cm}
%	\caption{Mean,. standard deviation and t tests for the average dwell time of each AOI across participant clusters. (FDR corrected)}
%	\label{tab:NetLossDwellByPID}
%\end{table}
%
%
%In Table \ref{tab:NetLossDwellByPID}, average dwell times across the four AOIs show no significant differences when compared between and within clusters of participant membership.
%
%\subsubsection{Number of Transitions}
%
% \begin{figure}[H]
%	\includegraphics[width=\linewidth]{../plots/PID/NTransitions.png}
%	\caption{Average number of transitions between clusters of participants determined from hierarchical clustering.}
%	%	\figurenote{+10 gain represents the proportion who chose to lie for all amounts above 10 and also applies for +20 and +30.}
%	\label{fig:NTransitionsByPIDCluster}
%\end{figure}
%
%The number of transitions averaged by the participants in Cluster 1 $(M = 5.1$, $SD = 3.9)$, showed to be significantly different to that of the participants in Cluster 2 $(M = 8.6\%$, $SD = 4.2\%)$, $t(112)=0.60$, $p=.070$. Comparatively, there was no significant difference in number of transitions within  Cluster 1 $t(378)=0.44$, $p=.0604$ but Cluster 2 $t(6)=3.5$, $p=.026$ there was found to be a significant difference (FDR corrected).
%
%\subsection{Cross Analysis}
%
%\subsubsection{Lie Percentage}
%
% \begin{figure}[H]
%	\includegraphics[width=\linewidth]{../plots/GainCluster/PercentLies.png}
%	\caption{Average lie percentage for each cluster of participants across trial conditions dependent on net gain to sender.}
%	%	\figurenote{+10 gain represents the proportion who chose to lie for all amounts above 10 and also applies for +20 and +30.}
%	\label{fig:PercentLiesPerGainByPIDCluster}
%\end{figure}
%
%\begin{table}[H]
%	\centering
%	\begin{tabular}{|c|p{1.5cm}|p{2cm}|p{1.5cm}|p{2cm}|p{2cm}|p{1.5cm}|}
%		\hline
%		\multirow{2}{*}{} & \multicolumn{2}{c|}{<10 Gain to Sender} & \multicolumn{2}{c|}{>10 Gain to Sender} & \multicolumn{2}{c|}{T-test result} \\ \cline{2-7}
%		& $M$ (\%) &$SD$ (\%) & $M$ (\%) & $SD$ (\%) & $t(66)$ & $p$ \\ \hline
%		Cluster 1& 20 & 40 & 56 & 49 & -0.26 & .793  \\ \hline
%		Cluster 2 & 34 & 47 & 52 & 50 & -0.74 & .460  \\ \hline
%	\end{tabular}
%	\vspace{0.3cm}
%	\caption{Comparison of mean and standard deviation of average percent lie for each cluster across trial conditions (FDR Corrected)}
%	\label{tab:PercentLiesPerGainByPIDCluster}
%\end{table}
%
%
%
%\subsubsection{Dwell Time}
%
% \begin{figure}[H]
%	\includegraphics[width=\linewidth]{figures/GainClusterTiles.png}
%	\caption{Measures of average dwell time across participant clusters segmented by hierarchical distance clustering.}
%	\label{fig:DwellTimesPerGainByPIDCluster}
%\end{figure}
%
%\begin{table}[H]
%	\centering
%	\begin{tabular}{|p{3.2cm}|p{0.7cm}|p{0.7cm}|p{0.7cm}|p{0.7cm}|p{0.9cm}|p{0.7cm}|p{0.7cm}|p{0.7cm}|p{0.8cm}|p{0.7cm}|}
%		\hline
%		\multirow{2}{*}{} & \multicolumn{4}{c|}{Gain to Sender < 10} & \multicolumn{4}{c|}{Gain to Sender > 10} & \multicolumn{2}{c|}{} \\ \cline{2-11}
%		\multirow{2}{*}{} & \multicolumn{2}{c|}{Cluster 1} & \multicolumn{2}{c|}{Cluster 2} & \multicolumn{2}{c|}{Cluster 1} & \multicolumn{2}{c|}{Cluster 2} & \multicolumn{2}{c|}{T-test result} \\ \cline{2-11}
%		& $M$ (ms) &$SD$ (ms) & $M$ (ms) & $SD$ (ms) & $M$ (ms) &$SD$ (ms) & $M$ (ms) & $SD$ (ms) & $t(34)$ & $p$ \\ \hline
%		SELF LIE& 0.49 & 0.38 & 0.24 & 0.34 & 0.49 & 0.38 & 0.24 & 0.34 & 1.9 & .126  \\ \hline
%		SELF TRUE & 0.44 & 0.30 & 0.23 & 0.33 & 0.49 & 0.38 & 0.24 & 0.34 & 1.5 & .081  \\ \hline
%		OTHER LIE & 0.51 & 0.32 & 0.39 & 0.49 & 0.49 & 0.38 & 0.24 & 0.34 & 0.19 & .211 \\ \hline
%		OTHER TRUTH & 0.54 & 0.35 & 0.40 & 0.48 & 0.49 & 0.38 & 0.24 & 0.34 & -0.15 & .184 \\ \hline
%	\end{tabular}
%	\vspace{0.3cm}
%	\caption{Mean and standard deviation of the dwell time of each AOI across participant clusters. }
%	\label{tab:NetGainDwellByPIDCluster}
%\end{table}
%
%
%\subsubsection{Number of Transitions}
%
% \begin{figure}[H]
%	\includegraphics[width=\linewidth]{../plots/GainCluster/NTransitions.png}
%	\caption{Measures of average dwell time across participant clusters segmented by hierarchical distance clustering.}
%	%	\figurenote{+10 gain represents the proportion who chose to lie for all amounts above 10 and also applies for +20 and +30.}
%	\label{fig:NTransitionPerGainByPIDCluster}
%\end{figure}





%The analyses are presented in this section. These analyses may be quantitative or qualitative. Feel free to use as many graphs, tables and diagrams as you think necessary for presenting the analyses clearly.  Do not comment on the meaning of the results here. Extrapolating from your results will happen in your Discussion section


\section{Discussion}

% Discussion which includes: summary of results/findings, understanding of results/findings, critical perspective and insight, indication of future research, organization and structure, and clarity

%In this section, you will summarise your project, expand upon your results, offer insights into strengths and limitations of your study and/or future directions.

\printbibliography

% References which includes correct use of APA/BPS conventions

\appendix

%This section should contain details which are not essential to understanding the essence of the report. Lists of stimuli, tables of raw data, details of statistical analyses should be included here. The section must be neatly presented and clearly labelled. If you have a very large amount of data, such as might arise in qualitative studies you should consult with your supervisor about what should be included and how.

\end{document}